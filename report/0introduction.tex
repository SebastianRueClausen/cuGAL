\section{Introduction}
Graphs are powerful and widely used data structures for modelling relationships between entities. They are employed to represent a wide range of systems, from road networks and social relations to biological systems and protein structures. Many algorithms have been developed to analyse and solve graph-related problems, among which is the problem of graph alignment. Graph alignment involves mapping nodes between two graphs to maximise a measure of similarity. This is often used to find matching entities in two graphs, such as the same persons in two social networks \citep{koutra2013big}, the same protein in two biological structures \citep{singh2008isorank} or image features in computer vision \citep{conte2003graph}. Due to the NP-hardness of the problem, algorithms will have to approximate an answer. To be applicable in a wide range of use cases, algorithms will also have to be resistant to noise, meaning no perfect mapping exists between the graphs.\\

In this project, we aim to improve the speed of an existing algorithm \textsc{Fugal} \citep{fugal2024} while maintaining its excellent accuracy compared to other algorithms. Although \textsc{Fugal} scales well compared to similar algorithms, performance remains a problem for larger graphs. We propose using GPUs (Graphical Processing Units) to take advantage of the parallelisable nature of the \textsc{Fugal} algorithm. We create an efficient GPU implementation of \textsc{Fugal} that takes advantage of modern GPU hardware. In addition, we make several modifications based on extensive analysis of \textsc{Fugal}, to further improve scalability for large graphs. We call our algorithm \textsc{cuGAL} - a combination of CUDA and \textsc{FUGAL}.\\

Throughout the project, we focus on different steps of the algorithm. We analyse each step and propose optimisations based on our findings. We then benchmark our implementation to test both accuracy and speed compared to \textsc{Fugal}.